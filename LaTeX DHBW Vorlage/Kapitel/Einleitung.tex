\section{Einleitung}
\label{sec:einleitung}
Die Frage nach der Mobilität der Zukunft ist eine aktuelle und häufig diskutierte Frage. Der Anwendungsfall bestimmt die geeignetste Antriebsart. Daher werden in Zukunft eine Vielzahl von Antriebsarten vorhanden sein. Die Ansätze reichen von der Entwicklung elektrischer Fahrzeuge mit verschiedenen Energiebereitstellungskonzepten bis hin zur Weiterentwicklung herkömmlicher Verbrennungsmotoren. Diese erfolgt simultan mit der Weiterentwicklung der Technologien zur Abgasreinigung. Durch die hohen Stickoxidemissionen älterer Systeme und die Verwendung rechtswidriger Abschaltvorrichtungen besteht im Allgemeinen eine Verunsicherung, wie effizient Dieselabgasanlagen sind \cite{Koch2018, Reif2010}. Im vergangenen Jahrzehnt wurde zur Erreichung der Abgasgrenzwerte ein \ac{DOC} zur Oxidation von Kohlenwasserstoffen (HC) und des Kohlenmonoxids (CO) verwendet. Außerdem sorgt ein \ac{DPF} für die Filterung von Rußpartikeln aus dem Abgas \cite{Reif2012}. Die \ac{AGR} erreicht durch entsprechende Abgasrückführraten eine Reduzierung usw. usw. usw.




